\section{Recursion in C Language}
\label{ch:recursion}

\subsection{Introduction to Recursion}
Recursion in computer science is a method where the solution to a problem depends on solutions to smaller instances of the same problem. In C language, a function can call itself if the solution of the problem can be obtained using smaller representations of the same problem.

\subsection{Basic Concept}
A recursive function is a function that solves a problem by solving smaller instances of the same problem. Below is the basic syntax of recursion in C language.

\begin{verbatim}
void recursiveFunction()
{
    // termination condition
    if(condition)
    {
        return;
    }
    else
    {
        recursiveFunction();
    }
}
\end{verbatim}

\subsection{Example of Recursion}
The classic example of recursion is the calculation of the factorial of a number. Here is how to implement it in C.

\begin{verbatim}
#include<stdio.h>

long long factorial(int n)
{
   if(n == 0)
      return 1;
   else
      return(n * factorial(n-1));
}

int main()
{
   int i;
   long long f;

   printf("Enter a number to compute its factorial\n");
   scanf("%d", &i);

   f = factorial(i);
   printf("%d! = %lld\n", i, f);

   return 0;
}
\end{verbatim}

\section{Exercises}

\subsection{Exercise 1}
Write a recursive function to calculate the sum of the first 'n' natural numbers.

% This could be a solution, but LaTeX does not support interactive click-to-show-the-solution buttons.
% \begin{verbatim}
% #include <stdio.h>
% int sum(int n)
% {
%     if (n != 0)
%         return n + sum(n - 1);
%     else
%         return n;
% }
% \end{verbatim}

\subsection{Exercise 2}
Write a recursive function to calculate the nth Fibonacci number.

% This could be a solution, but LaTeX does not support interactive click-to-show-the-solution buttons.
% \begin{verbatim}
% #include <stdio.h>
% int fibonacci(int n)
% {
%     if(n == 0)
%         return 0;
%     else if(n == 1)
%         return 1;
%     else
%         return(fibonacci(n-1) + fibonacci(n-2));
% }
% \end{verbatim}

