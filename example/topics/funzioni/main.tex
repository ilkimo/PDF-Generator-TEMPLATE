\section{Funzioni}
\label{ch:funzioni}

\subsection{Introduzione alle funzioni}
Immagina di avere un robot in cucina che prepara i tuoi piatti preferiti. Gli dai la ricetta (che potrebbe essere fare un toast, preparare un caffè, o cucinare una lasagna), e lui eseguirà esattamente ciò che gli hai chiesto di fare. In programmazione, una \textbf{funzione} è come quel robot. Le funzioni sono blocchi di codice che compiono un'azione specifica quando chiamati nel tuo programma.

\subsubsection{Esempio di funzione}
Ecco un esempio di una funzione in C che stampa "Ciao, mondo!".

\needspace{10\baselineskip}
\begin{lstlisting}[style=codeSnippet]
#include <stdio.h>

void saluta() {
   printf("Ciao, mondo!\n");
}

int main() {
   saluta(); // questo chiama la funzione saluta
   return 0;
}
\end{lstlisting}
