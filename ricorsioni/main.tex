\section{Ricorsioni}
\label{ch:ricorsioni}

\subsection{Introduzione alle Ricorsioni}
% La ricorsione in informatica è un metodo in cui la soluzione di un problema dipende 
% dalle soluzioni di istanze più piccole dello stesso problema. Nel linguaggio C, una 
% funzione può chiamare se stessa se la soluzione del problema può essere ottenuta usando 
% rappresentazioni più piccole dello stesso problema.

La ricorsione è un concetto molto importante nell'informatica e può essere paragonata ad 
una serie di scatole nidificate, o ad un gioco di specchi che riflettono l'uno nell'altro 
all'infinito (queste analogie risultano più chiare quando si vede lo stack di chiamate di 
un programma!). Supponiamo che tu stia cercando un libro in una pila di libri. 
Il tuo approccio potrebbe 
essere il seguente:
\begin{enumerate}
	\item Guardi il libro in cima alla pila. 
	\item Se il libro che stai cercando è quello, l'hai trovato! Se non è quello, metti da 
		parte il libro e ripeti il processo con la pila di libri rimanente. 
\end{enumerate}
Questa è una forma di ricorsione. 
Stai ripetendo lo stesso processo (cercare il libro in cima alla pila) finché non 
trovi quello che stai cercando o finché non hai esaurito i libri. 
In termini di programmazione, una funzione ricorsiva è una funzione che chiama se 
stessa nel suo codice, semplificando di volta in volta il problema (riducendo 
la pila di libri), finché non risulterà talmente semplice da essere già risolto 
(quando la pila ha solo più un libro, se è quello che stavi cercando, allora l'hai trovato, 
se non è quello, allora non c'è nella pila)!

\subsection{Concetto di Base}
Una funzione ricorsiva è una funzione che risolve un problema risolvendo istanze 
più piccole dello stesso problema. Di seguito è riportata la sintassi di base della 
ricorsione nel linguaggio C.

\needspace{10\baselineskip}
\begin{lstlisting}[style=codeSnippet]
void funzioneRicorsiva(<qualche argomento>) {
	if(<condizione>) {
		// caso base
		return <risultato banale>;
	} else {
		// caso ricorsivo
		<qualche operazione>;
		return funzioneRicorsiva(<altro argomento>);
	}
}
\end{lstlisting}

\subsection{Esempio di Ricorsione}
L'esempio classico di ricorsione è il calcolo del fattoriale di un numero. Ecco come implementarlo in C.

\begin{verbatim}
#include<stdio.h>

long long fattoriale(int n)
{
   if(n == 0)
	  return 1;
   else
	  return(n * fattoriale(n-1));
}

int main()
{
   int i;
   long long f;

   printf("Inserire un numero per calcolare il suo fattoriale\n");
   scanf("%d", &i);

   f = fattoriale(i);
   printf("%d! = %lld\n", i, f);

   return 0;
}
\end{verbatim}

\section{Esercizi}

\subsection{Esercizio 1}
Scrivi una funzione ricorsiva per calcolare la somma dei primi 'n' numeri naturali.

% Questa potrebbe essere una soluzione, ma LaTeX non supporta pulsanti interattivi click-to-show-the-solution.
% \begin{verbatim}
% #include <stdio.h>
% int somma(int n)
% {
%     if (n != 0)
%         return n + somma(n - 1);
%     else
%         return n;
% }
% \end{verbatim}

\subsection{Esercizio 2}
Scrivi una funzione ricorsiva per calcolare l'n-esimo numero di Fibonacci.

% Questa potrebbe essere una soluzione, ma LaTeX non supporta pulsanti interattivi click-to-show-the-solution.
% \begin{verbatim}
% #include <stdio.h>
% int fibonacci(int n)
% {
%     if(n == 0)
%         return 0;
%     else if(n == 1)
%         return 1;
%     else
%         return(fibonacci(n-1) + fibonacci(n-2));
% }
% \end{verbatim}

